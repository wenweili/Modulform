% Copyright 2020  李文威 (Wen-Wei Li).
% Permission is granted to copy, distribute and/or modify this
% document under the terms of the Creative Commons
% Attribution 4.0 International (CC BY 4.0)
% http://creativecommons.org/licenses/by/4.0/

% 目的: 设置章节标题格式及目录的显式方式.
% 由 AJbook.cls 引入
\ProvidesFile{titles-setup.tex}[2018/03/03]

\RequirePackage[calcwidth, nobottomtitles, explicit, newparttoc, indentafter]{titlesec}      % 标题格式: explicit 选项导致须在 titleformat 的 before-code 中加 #1. 选项 newparttoc 用来将各部分加入目录. indentafter: 首行一律缩进
\RequirePackage{titletoc}                 % 目录格式

% 目录部分: 章名除附录外仍用中文标号, 黑体显示. 以参数是否为大写拉丁字母来判定是否在附录 (烂招)
\if@AJ@CJKthechapter
	\providecommand{\AJchapterttl}[1]{\IfSubStr{ABCDEFGHIJKLMNOPQRSTUVWXYZ}{#1}{附录 #1	}{第\CJKnumber{#1}章}}
\else
	\providecommand{\AJchapterttl}[1]{\IfSubStr{ABCDEFGHIJKLMNOPQRSTUVWXYZ}{#1}{附录 #1	}{第 #1 章}}
\fi

\newlength{\BoxTtlwidth}	% 用来计算各种盒子所需宽度

% 各章标题排版
\titleformat{\chapter}
	{\filleft\normalfont\Huge\bfseries\mathversion{bold}\CJKfamily{chapterfont}}
	{\AJchapterttl{\thechapter}}
	{5mm}
	{#1}
	[{\vspace{2em} \titleline[c]{\titlerule[1pt]}}]

\titlespacing*{\chapter}{1pc}{*4}{5em}

\titleformat{name=\chapter, numberless}
	{\filleft\normalfont\Huge\bfseries\mathversion{bold}\CJKfamily{chapterfont}}	% Format
	{}	% Label
	{5mm}	% Sep
	{#1}	% Before-code
	[{\vspace{2em} \titleline[c]{\titlerule[1pt]}
	\if@mainmatter
		\addcontentsline{toc}{chapter}{#1}
		\markboth{#1}{}
	\fi
	}]	% After-code: 无号章如果出现在正文中, 就加入目录并相应地设置天眉.

\titlespacing*{name=\chapter, numberless}	% 设置间隔
{1pc}{*4}{1em}	% {left}{before-sep}{after-sep}

\titleformat{\section}
	{\filright\bfseries\mathversion{bold}\Large\sffamily\CJKfamily{sectionfont}}
	{	% 烂招: 直接将整个标题插入为 label
		\settowidth{\BoxTtlwidth}{\Huge \thesection \hspace{0.7em} \Large #1}  % 首先计算宽度
		\ifdim \BoxTtlwidth < \textwidth % 一般情形下调用 \MakeSectBox
			\MakeSectBox{\Huge \thesection \hspace{0.7em} \Large #1} %
		\else % 万一标题过长则改用 minipage 以确保正常断行 (烂招)
			\begin{minipage}[c]{\textwidth} %
				\Huge \underline{\thesection} \hspace{0.7em} \Large #1 %
			\end{minipage} %
		\fi %
	}
	{0.7em}
	{}
	[]
\titlespacing*{\section}{1pc}{*1.3}{*1}    % \titlespacing{command}{left}{before-sep}{after-sep}

\titleformat{name=\section, numberless}
	{\filleft\bfseries\mathversion{bold}\Large\sffamily\CJKfamily{sectionfont}}
	{	% 烂招: 直接将整个标题插入为 label
		\settowidth{\BoxTtlwidth}{\Large #1}  % 首先计算宽度
		\ifdim \BoxTtlwidth < \textwidth % 一般情形下调用 \MakeSectBox
			\MakeSectBox{\Large #1} %
		\else % 万一标题过长则改用 minipage 以确保正常断行 (烂招)
			\begin{minipage}[c]{\textwidth} %
				\Large #1 %
			\end{minipage} %
		\fi %
	}
	{0mm}
	{}
	[]

\titleformat{name=\subsection} % 各子节标题, 采取 runin 形式较美观
	[runin]
	{\filleft\normalfont\sffamily\bfseries\mathversion{bold}\CJKfamily{sectionfont}}
	{\Large\thesubsection}	% Label
	{3mm}	% Sep
	{#1}	% Before-code
	[]		% After-code

\titleformat{name=\subsection, numberless}
	[runin]
	{\filleft\normalfont\sffamily\bfseries\mathversion{bold}\CJKfamily{sectionfont}}
	{}	% Label
	{0mm}	% Sep
	{#1}	% Before-code
	[]		% After-code

\titlespacing*{name=\subsection}	% 设置间隔
	{0pt}{*1}{1em}    % {left}{before-sep}{after-sep}

\titleformat{name=\subsubsection} % 次子节标题, 采取 runin 形式较美观
	[runin]
	{\filleft\normalfont\sffamily\bfseries\mathversion{bold}\CJKfamily{sectionfont}}
	{\thesubsubsection}	% Label
	{3mm}	% Sep
	{#1}	% Before-code
	[]		% After-code

\titleformat{name=\subsubsection, numberless}
	[runin]
	{\filleft\normalfont\sffamily\bfseries\mathversion{bold}\CJKfamily{sectionfont}}
	{}	% Label
	{0mm}	% Sep
	{#1}	% Before-code
	[]		% After-code

\titlespacing*{name=\subsubsection}	% 设置间隔
	{0pt}{*1}{1em}    % {left}{before-sep}{after-sep}

% 不应重新定义 \thepart 命令, 否则 PDF bookmark 将有问题. 改为在 Label 部份转换中文数字.
\titleformat{name=\part}[display]	% 各部分标题
	{\filcenter\sffamily\bfseries\mathversion{bold}\CJKfamily{song}\Huge}	% Format
	{第{\CJKnumber{\arabic{part}}}部分}	% Label
	{1.5em}	% Sep
	{#1}	% Before-code
	[]	% After-code

% 各节标题排版: \MakeSectBox{文字}
\newtcbox{\MakeSectBox}{
	enhanced,
	arc = 0pt, outer arc = 0pt,
	before skip = 0pt, after skip = 0.4em, left skip = 0pt, right skip = 0pt,
	top = 10pt, left = 0pt, right = 0pt, bottom = 1.5mm,
	sharp corners = all,
	colback = white,
	colframe = white,
	boxsep = 0pt, leftrule = 0pt, rightrule=0pt, toprule=0pt,
	bottomrule = 0pt,
	valign=bottom,
	overlay = { \draw[line width=1pt] (interior.south west) -- (interior.south east); }
}

\titlecontents{chapter}
	[0pt]
	{\addvspace{1pc}\heiti}
	{\contentsmargin{0pt}\large\AJchapterttl{\thecontentslabel} \quad}
	{\contentsmargin{0pt}\large}
	{\titlerule*[.7pc]{.}\contentspage}

\titlecontents{section}
	[1.5em]
	{}
	{\thecontentslabel\quad}
	{\thecontentslabel}
	{\titlerule*[.7pc]{.}\contentspage}

\titlecontents{subsection}
	[3.9em]
	{\small}
	{\thecontentslabel\quad}
	{\ (\thecontentspage)}
	{\titlerule*[.7pc]{.}\contentspage}

\titlecontents{part}
	[0cm]
	{\addvspace{1pc}\songti}
	{\contentsmargin{0pt} \large 第{\thecontentslabel}部分\quad}
	{\large}
	{\titlerule*[.7pc]{.}\contentspage}
