%!TEX TS-program = xelatex
%!TEX encoding = UTF-8

% LaTeX source for the errata of the book ``模形式初步'' in Chinese
% Copyright 2023  李文威 (Wen-Wei Li).
% Permission is granted to copy, distribute and/or modify this
% document under the terms of the Creative Commons
% Attribution 4.0 International (CC BY 4.0)
% http://creativecommons.org/licenses/by/4.0/

% 《模形式初步》勘误表 / 李文威
% 使用自定义的文档类 AJerrata.cls. 自动载入 xeCJK.

\documentclass{AJerrata}

\usepackage{unicode-math}

\usepackage[unicode, colorlinks, psdextra, bookmarksnumbered,
	pdfpagelabels=true,
	pdfauthor={李文威 (Wen-Wei Li)},
	pdftitle={模形式初步勘误},
	pdfkeywords={}
]{hyperref}

\setmainfont[
	BoldFont={texgyretermes-bold.otf},
	ItalicFont={texgyretermes-italic.otf},
	BoldItalicFont={texgyretermes-bolditalic.otf},
	PunctuationSpace=2
]{texgyretermes-regular.otf}

\setsansfont[
	BoldFont=FiraSans-Bold.otf,
	ItalicFont=FiraSans-Italic.otf
]{FiraSans-Regular.otf}

\setCJKmainfont[
	BoldFont=Noto Serif CJK SC Bold
]{Noto Serif CJK SC}

\setCJKsansfont[
	BoldFont=Noto Sans CJK SC Bold
]{Noto Sans CJK SC}

\setCJKfamilyfont{emfont}[
	BoldFont=FandolHei-Regular.otf
]{FandolHei-Regular.otf}	% 强调用的字体

\renewcommand{\em}{\bfseries\CJKfamily{emfont}} % 强调

\setmathfont[
	Extension = .otf,
	math-style= TeX,
]{texgyretermes-math}

\usepackage{mathrsfs}
\usepackage{stmaryrd} \SetSymbolFont{stmry}{bold}{U}{stmry}{m}{n}	% 避免警告 (stmryd 不含粗体故)
% \usepackage{array}
% \usepackage{tikz-cd}  % 使用 TikZ 绘图
\usetikzlibrary{positioning, patterns, calc, matrix, shapes.arrows, shapes.symbols}

\usepackage{myarrows}				% 使用自定义的可伸缩箭头
\usepackage{mycommand}				% 引入自定义的惯用的命令

\newcommand{\bomega}{\symbf{\omega}}	% Boldface omega, for sheaves of differentials

\title{\bfseries 《模形式初步》勘误表 \\ 跨度: 2022 修订版迄今}
\author{李文威}
\date{\today}

\begin{document}
	\maketitle
	
	\begin{Errata}
		\item[第 2 页第一行 (仅 PDF 版)]
		\Orig 透过过
		\Corr 透过
		
		\item[导言的拓扑空间符号部分]
		\Orig $\partial D := D \smallsetminus D^\circ$
		\Corr $\partial D := \overline{D} \smallsetminus D^\circ$
		\Thx{感谢雷嘉乐指正}
		
		\item[导言的矩阵符号部分中部]
		\Orig $\Image[\SL(n, R) \to \PGL(n, \R)]$
		\Corr $\Image[\SL(n, R) \to \PGL(n, R)]$
		\Thx{感谢雷嘉乐指正}
		
		\item[\S 1.1 第一个脚注 (仅纸本)]
		\Orig [50]
		\Corr [59]
		\Thx{感谢孙超超指正}
		
		\item[命题 1.4.12 关于 $\Stab_{\SL(2, \Z)}(\rho)$ 生成元的描述]
		\Orig $\bigl(\begin{smallmatrix} & -1 \\ 1 & -1 \end{smallmatrix}\bigr)$
		\Corr $\bigl(\begin{smallmatrix} & 1 \\ -1 & 1 \end{smallmatrix}\bigr)$
		\Thx{感谢余君指正}
		
		\item[引理 2.1.5 证明倒数第二行]
		\Orig $n^!$
		\Corr $n!$
		\Thx{感谢 Wenjun Huang 指正}

		\item[例 3.5.4 之前的 (i)]
		\Orig 线性代群
		\Corr 线性代数群
		\Thx{感谢杨箐浩指正}


		\item[定义 3.6.4 之后的讨论条列第二项]
		从``变 $\alpha$ 为 $\alpha\beta$...'' 之后关于 $g^*$ 的公式起, 直到 ``... 只差一个因子 $a^{-k}$.'' 为止, 所有的 $a^{-k}$ 都应该改成 $a^k$ (共 5 处)
		\Thx{感谢余君指正}
		
		\item[定理 5.2.7 证明第一段最末]
		\Orig 所以 $\gamma \in \Gamma$
		\Corr 所以 $\gamma' \in \Gamma$
		\Thx{感谢张羽扬指正}
		
		\item[等式 (5.2.1) 的下一行]
		\Orig $\Gamma \cdot \Gamma \alpha \Gamma$
		\Corr $\Gamma' \cdot \Gamma \alpha \Gamma$
		\Thx{感谢汤一鸣指正}
		
		\item[等式 (5.4.1) 的下一行]
		\Orig $f(\delta_1 \delta_1)$
		\Corr $f(\delta_1 \delta_2)$
		\Thx{感谢汤一鸣指正}
		
		\item[公式 (6.2.3)]
		将两处 $L/L'$ 改成 $L'/L$.
		
		\Thx{感谢张羽扬指正}
		
		\item[定理 6.5.1 证明]
		将证明中间``定义 $S_k(\Gamma(N))$ 的线性自同态...''之前一行的显示公式中的所有 $\alpha_n(f)$ 改为 $\alpha_n(\varphi)$.
		
		\Thx{感谢余君指正}
		
		\item[定理 7.1.2 证明第一行]
		在``命题 2.6.3''之后加上一条脚注: ``对于该节构造的 Eisenstein 级数, 其 Fourier 展开和对应的直和分解可以通过解析延拓推及 $k = 1, 2$ 的情形, 细节比较复杂, 详阅 [41, \S 7.2].''
		
		\Thx{感谢彭也博指正}
		
		\item[(9.1.5) 之下第三行]
		\Orig $\in \Gamma(\mathcal{H}, \bomega)$
		\Corr $\in \Gamma(\mathcal{H}, \bomega^{\otimes k})$
		
		\item[练习 10.1.3 之前一行]
		\Orig ...有奇点
		\Corr ...无奇点
		\Thx{感谢刘亚迪指正}
		
		\item[引理 A.1.2 之前两行]
		\Orig ... 连续的最粗拓扑
		\Corr ... 连续的最细拓扑
		\Thx{感谢李钦浩指正}
		
		\item[引理 A.1.10 证明第三行]
		\Orig $G/K$
		\Corr $G/H$
	\end{Errata}
\end{document}
